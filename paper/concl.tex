\section{Conclusion}
\label{sec:concl}

The Internet is an archetypical example of the ``Worse is Better'' approach to software design: its versatility and ubiquity are due in large part to the simplicity of its basic design, and the conscious decisions made by its designers to not provide certain guarantees or services to users of the network. However, the concept of resource pooling, as elaborated by Wischik et al. \cite{wischik2008resource}, is so important and so central to providing reliable performance that it has been implemented over and over again by various stakeholders in the Internet ecosystem, often in ways that adversely affect the network at large or impose extra costs and redundant design. A multipath transport protocol provides a way to implement resource pooling in a consistent, elegant way that will benefit users, administrators, and designers across the Internet. Although MPTCP is an excellent step towards a multipath transport protocol, its design is constrained by the need to accomodate middleboxes and a network architecture not designed for multipath. A simpler and more flexible protocol can be created by making use of the versatile network abstraction provided by Serval.

My protocol design is based around guiding principle of ``Dumb Subflows, Smart Connection'': as much state and complexity should be pushed to the connection level, to avoid the redundant computations and lack of flexibility that come from putting too much complexity at the subflow level. Compared to MPTCP, this allows a simpler congestion control mechanism and more flexibility in retransmissions and acknowledgements, as well as the ability to quickly adjust to changing network conditions. Connection state is managed through a segment table which tracks byte ranges sent out over the various subflows, and is continually updated based on the SACKs received. A proof of concept implementation has shown that the segment table data structure and selective acknowledgment system described above works, and progress has been made towards a working implementation within the Linux kernel.

